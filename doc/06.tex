\documentclass{article}

\usepackage{amssymb, amsmath}
\usepackage{alltt}
\usepackage{pslatex}
\usepackage{epigraph}
\usepackage{verbatim}
\usepackage{latexsym}
\usepackage{array}
\usepackage{comment}
\usepackage{makeidx}
\usepackage{listings}
\usepackage{indentfirst}
\usepackage{verbatim}
\usepackage{color}
\usepackage{url}
\usepackage{xspace}
\usepackage{hyperref}
\usepackage{stmaryrd}
\usepackage{amsmath, amsthm, amssymb}
\usepackage{graphicx}
\usepackage{euscript}
\usepackage{mathtools}
\usepackage{mathrsfs}
\usepackage{multirow,bigdelim}
\usepackage{subcaption}
\usepackage{placeins}

\makeatletter

\makeatother

\definecolor{shadecolor}{gray}{1.00}
\definecolor{darkgray}{gray}{0.30}

\def\transarrow{\xrightarrow}
\newcommand{\setarrow}[1]{\def\transarrow{#1}}

\def\padding{\phantom{X}}
\newcommand{\setpadding}[1]{\def\padding{#1}}

<<<<<<< HEAD
\def\subarrow{}
\newcommand{\setsubarrow}[1]{\def\subarrow{#1}}

\newcommand{\trule}[2]{\frac{#1}{#2}}
\newcommand{\crule}[3]{\frac{#1}{#2},\;{#3}}
\newcommand{\withenv}[2]{{#1}\vdash{#2}}
\newcommand{\trans}[3]{{#1}\transarrow{\padding#2\padding}\subarrow{#3}}
\newcommand{\ctrans}[4]{{#1}\transarrow{\padding#2\padding}\subarrow{#3},\;{#4}}
=======
\newcommand{\trule}[2]{\frac{#1}{#2}}
\newcommand{\crule}[3]{\frac{#1}{#2},\;{#3}}
\newcommand{\withenv}[2]{{#1}\vdash{#2}}
\newcommand{\trans}[3]{{#1}\transarrow{\padding#2\padding}{#3}}
\newcommand{\ctrans}[4]{{#1}\transarrow{\padding#2\padding}{#3},\;{#4}}
>>>>>>> hw7
\newcommand{\llang}[1]{\mbox{\lstinline[mathescape]|#1|}}
\newcommand{\pair}[2]{\inbr{{#1}\mid{#2}}}
\newcommand{\inbr}[1]{\left<{#1}\right>}
\newcommand{\highlight}[1]{\color{red}{#1}}
\newcommand{\ruleno}[1]{\eqno[\scriptsize\textsc{#1}]}
\newcommand{\rulename}[1]{\textsc{#1}}
\newcommand{\inmath}[1]{\mbox{$#1$}}
\newcommand{\lfp}[1]{fix_{#1}}
\newcommand{\gfp}[1]{Fix_{#1}}
\newcommand{\vsep}{\vspace{-2mm}}
\newcommand{\supp}[1]{\scriptsize{#1}}
\newcommand{\sembr}[1]{\llbracket{#1}\rrbracket}
\newcommand{\cd}[1]{\texttt{#1}}
\newcommand{\free}[1]{\boxed{#1}}
\newcommand{\binds}{\;\mapsto\;}
\newcommand{\dbi}[1]{\mbox{\bf{#1}}}
\newcommand{\sv}[1]{\mbox{\textbf{#1}}}
\newcommand{\bnd}[2]{{#1}\mkern-9mu\binds\mkern-9mu{#2}}
\newtheorem{lemma}{Lemma}
\newtheorem{theorem}{Theorem}
\newcommand{\meta}[1]{{\mathcal{#1}}}
\renewcommand{\emptyset}{\varnothing}

\definecolor{light-gray}{gray}{0.90}
\newcommand{\graybox}[1]{\colorbox{light-gray}{#1}}

\lstdefinelanguage{ocaml}{
keywords={let, begin, end, in, match, type, and, fun, 
function, try, with, class, object, method, of, rec, repeat, until,
while, not, do, done, as, val, inherit, module, sig, @type, struct, 
<<<<<<< HEAD
if, then, else, open, virtual, new, fresh, skip, od, fi, elif, for, local, return, read, write},
=======
if, then, else, open, virtual, new, fresh, skip, od, fi, elif, for, local},
>>>>>>> hw7
sensitive=true,
%basicstyle=\small,
commentstyle=\scriptsize\rmfamily,
keywordstyle=\ttfamily\bfseries,
identifierstyle=\ttfamily,
basewidth={0.5em,0.5em},
columns=fixed,
fontadjust=true,
literate={->}{{$\to$}}3 {===}{{$\equiv$}}1 {=/=}{{$\not\equiv$}}1 {|>}{{$\triangleright$}}3 {\&\&\&}{{$\wedge$}}2 {|||}{{$\vee$}}2 {^}{{$\uparrow$}}1,
morecomment=[s]{(*}{*)}
}

\lstset{
mathescape=true,
%basicstyle=\small,
identifierstyle=\ttfamily,
keywordstyle=\bfseries,
commentstyle=\scriptsize\rmfamily,
basewidth={0.5em,0.5em},
fontadjust=true,
escapechar=!,
language=ocaml
}

\sloppy

\newcommand{\ocaml}{\texttt{OCaml}\xspace}

\theoremstyle{definition}

<<<<<<< HEAD
\title{Functions\\
=======
\title{Procedures\\
>>>>>>> hw7
  (the first draft)
}

\author{Dmitry Boulytchev}

\begin{document}

\maketitle

<<<<<<< HEAD
\section{Functions in Expressions}

At the syntax level function calls are introduced with the following construct: 

\[
\begin{array}{rcl}
  \mathscr E & += & \mathscr X \mathscr E^*
\end{array}
\]

In the concrete syntax function call looks conventional:

\begin{lstlisting}
     $f$ ($e_1, e_2, \dots, e_k$)
\end{lstlisting}

where $f$~--- a function name, $e_i$~--- its actual parameters.

Surpisingly, such a mild extension results in a complete redefinition of the semantics. Indeed, as function body can perform
arbitrary actions on state, input and output streams, expressions with function calls can now have side effects. In order to express these
side effects we need to redefine the semantics completely, this time using big-step operational style.

We extend the configuration, used in the semantic description for statements, with a fourth component~--- an optional integer value:

\[
\mathscr C = \Sigma \times \mathbb Z^* \times \mathbb Z^* \times \mathbb Z^?
\]

This component will correspond to an optional return value for function/procedure calls (either integer value $n$ or ``$\hbox{---}$'', nothing).

The rules inself are summarized in the Fig.~\ref{bs_expr}. Note the use of double environment for evaluating the body of function in the rule
\rulename{Call}; note also, that now semantics of expressions and statements are mutually recursive.

\setarrow{\xRightarrow}
\setsubarrow{_{\mathscr E}}

\begin{figure}
  \[\withenv{\Phi}{\trans{\inbr{\sigma, i, o, \hbox{---}}}{n}{\inbr{\sigma, i, o, n}}} \ruleno{Const} \]
  \[\withenv{\Phi}{\trans{\inbr{\sigma, i, o, \hbox{---}}}{x}{\inbr{\sigma, i, o, \sigma\;x}}} \ruleno{Var} \]
  \[\trule{\withenv{\Phi}{\trans{c}{A}{c^\prime=\inbr{\_,\, \_,\, \_,\, a}}},\;
           \withenv{\Phi}{\trans{c^\prime}{B}{\inbr{\sigma^{\prime\prime}, i^{\prime\prime}, o^{\prime\prime}, b}}}
          }
          {\withenv{\Phi}{\trans{c}{A\otimes B}{\inbr{\sigma^{\prime\prime}, i^{\prime\prime}, o^{\prime\prime}, a\oplus b}}}}
          \ruleno{Binop}
  \]
  \[\trule{\begin{array}{c}
             \withenv{\Phi}{\trans{c_{j-1}}{e_j}{c_j=\inbr{\sigma_j,\,i_j,\,o_j,\,v_j}}},\\
             \Phi\;f = \llang{fun $\;f\;$ ($\overline{a}$) local $\;\overline{l}\;$ \{$s$\}},\\
             {\setsubarrow{}
             \withenv{\llang{skip},\,\Phi}{\trans{\inbr{\mbox{\textbf{enter}}\;\sigma_k\; (\overline{a}@\overline{l})\; [\overline{a_j\gets v_j}],i_k,o_k,\hbox{---}}}{s}{\inbr{\sigma^\prime, i^\prime, o^\prime, n}}}}
           \end{array}
          }
          {\withenv{\Phi}{\trans{c_0=\inbr{\sigma_0,\,\_,\,\_,\,\_}}{f (\overline{e_k})}{\inbr{\mbox{\textbf{leave}}\;\sigma^\prime\;\sigma_0, i^\prime, o^\prime, n}}}}
          \ruleno{Call}
  \]
\caption{Big-step Operational Semantics for Expressions}
\label{bs_expr}
\end{figure}

\section{Return Statement}

In order to make it possible to return values from procedures we add a return statement to the language of statements:

\[
\mathscr S += \llang{return $\;\mathscr E^?$}
\]

And, again, this small addition leads to redefinition of the semantics in \emph{continuation-passing style} (CPS).

First, we define the following meta-operator ``$\diamond$'' on statements:

\[
\begin{array}{rcl}
  s \diamond \llang{skip} & = & s\\
  s_1 \diamond s_2 & = & \llang{$s_1$; $\;s_2$}
\end{array}
\]

\setsubarrow{}

Then, we add another environment component, $K$, in the description of semantic relation ``$\withenv{...}{\trans{...}{...}{...}}$''. Informally
speaking, now

\[
\withenv{K,\,\Phi}{\trans{c}{s}{c^\prime}}
\]

is read ``the execution of $s$, immediately followed by $K$, in the configuration $c$ results in the configuration $c^\prime$''. Statement $K$ is called \emph{continuation}.
The rules themselves are shown on Fig.~\ref{bs_cps}. Note, the rule for the call statement is exactly the same, as for the call expression.

\begin{figure}
  \[\withenv{\llang{skip},\,\Phi}{\trans{c}{\llang{skip}}{c}}\ruleno{SkipSkip}\]
  \[\trule{\withenv{\llang{skip},\,\Phi}{\trans{c}{K}{c^\prime}}}
          {\withenv{K,\,\Phi}{\trans{c}{\llang{skip}}{c^\prime}}}
    \ruleno{Skip}
  \]
  \[\trule{{\setsubarrow{_{\mathscr E}}\withenv{\Phi}{\trans{c}{e}{\inbr{\sigma,\,i,\,o,\,n}}}};\;
           \withenv{\llang{skip},\,\Phi}{\trans{\inbr{\sigma[x\gets n],\,i,\,o,\,\hbox{---}}}{K}{c^\prime}}
          }
          {\withenv{K,\,\Phi}{\trans{c}{\llang{$x\,$:=$\,e$}}{c^\prime}}}
    \ruleno{Assign}
  \]
  \[\trule{{\setsubarrow{_{\mathscr E}}\withenv{\Phi}{\trans{c}{e}{\inbr{\sigma,\,i,\,o,\,n}}}};\;
           \withenv{\llang{skip},\,\Phi}{\trans{\inbr{\sigma,\,i,\,o@[n],\,\hbox{---}}}{K}{c^\prime}}
          }
          {\withenv{K,\,\Phi}{\trans{c}{\llang{write ($e$)}}{c^\prime}}}
    \ruleno{Write}
  \]
  \[\trule{
           \withenv{\llang{skip},\,\Phi}{\trans{\inbr{\sigma[x\gets z],\,i,\,o,\,\hbox{---}}}{K}{c^\prime}}
          }
          {\withenv{K,\,\Phi}{\trans{\inbr{\sigma,\,z:i,\,o,\,\hbox{---}}}{\llang{read ($x$)}}{c^\prime}}}
    \ruleno{Read}
  \]

  \[\trule{\withenv{s_2\diamond K,\,\Phi}{\trans{c}{\llang{$s_1$}}{c^\prime}}}
          {\withenv{K,\,\Phi}{\trans{c}{\llang{$s_1$; $\;s_2$}}{c^\prime}}}
    \ruleno{Seq}
  \]

  \[\crule{{\setsubarrow{_{\mathscr E}} \withenv{\Phi}{\trans{c}{e}{\inbr{\sigma,\,i,\,o,\,n}}}};\;
           \withenv{K,\,\Phi}{\trans{\inbr{\sigma,\,i,\,o,\,\hbox{---}}}{\llang{$s_1$}}{c^\prime}}}
          {\withenv{K,\,\Phi}{\trans{c}{\llang{if $\;e\;$ then $\;s_1\;$ else $\;s_2$}}{c^\prime}}}
          {n\ne 0}
    \ruleno{IfTrue}
  \]

  \[\crule{{\setsubarrow{_{\mathscr E}} \withenv{\Phi}{\trans{c}{e}{\inbr{\sigma,\,i,\,o,\,n}}}};\;
           \withenv{K,\,\Phi}{\trans{\inbr{\sigma,\,i,\,o,\,\hbox{---}}}{\llang{$s_2$}}{c^\prime}}}
          {\withenv{K,\,\Phi}{\trans{c}{\llang{if $\;e\;$ then $\;s_1\;$ else $\;s_2$}}{c^\prime}}}
          {n= 0}
    \ruleno{IfFalse}
  \]

  \[\crule{\begin{array}{c}
             {\setsubarrow{_{\mathscr E}} \withenv{\Phi}{\trans{c}{e}{\inbr{\sigma,\,i,\,o,\,n}}}}\\
             \withenv{\llang{while $\;e\;$ do $\;s$}\diamond K,\,\Phi}{\trans{\inbr{\sigma,\,i,\,o,\,\hbox{---}}}{\llang{$s$}}{c^\prime}}
           \end{array}}
          {\withenv{K,\,\Phi}{\trans{c}{\llang{while $\;e\;$ do $\;s$}}{c^\prime}}}
          {n\ne 0}
    \ruleno{WhileTrue}
  \]

    \[\crule{{\setsubarrow{_{\mathscr E}} \withenv{\Phi}{\trans{c}{e}{\inbr{\sigma,\,i,\,o,\,n}}}};\;
           \withenv{\llang{skip},\,\Phi}{\trans{\inbr{\sigma,\,i,\,o,\,\hbox{---}}}{K}{c^\prime}}}
          {\withenv{K,\,\Phi}{\trans{c}{\llang{while $\;e\;$ do $\;s$}}{c^\prime}}}
          {n = 0}
    \ruleno{WhileFalse}
  \]

  \[\trule{\begin{array}{c}
             {\setsubarrow{_{\mathscr E}}\withenv{\Phi}{\trans{c_{j-1}}{e_j}{c_j=\inbr{\sigma_j,\,i_j,\,o_j,\,v_j}}}}\\
             \Phi\;f = \llang{fun $\;f\;$ ($\overline{a}$) local $\;\overline{l}\;$ \{$s$\}}\\
             \withenv{\llang{skip},\,\Phi}{\trans{\inbr{\mbox{\textbf{enter}}\;\sigma_k\; (\overline{a}@\overline{l})\; [\overline{a_j\gets v_j}],i_k,o_k,\hbox{---}}}{s}{\inbr{\sigma^\prime, i^\prime, o^\prime, n}}}
           \end{array}
          }
          {\withenv{K,\,\Phi}{\trans{c_0=\inbr{\sigma_0,\,\_,\,\_,\,\_}}{f (\overline{e_k})}{\inbr{\mbox{\textbf{leave}}\;\sigma^\prime\;\sigma_0, i^\prime, o^\prime, n}}}}
          \ruleno{Call}
  \]
  
  \[
    \withenv{K,\,\Phi}{\trans{c}{\llang{return}}{c}}
    \ruleno{ReturnEmpty}
  \]

  \[
    \trule{{\setsubarrow{_{\mathscr E}} \withenv{\Phi}{\trans{c}{e}{c^\prime}}}}
          {\withenv{K,\,\Phi}{\trans{c}{\llang{return $\;e\;$}}{c^\prime}}}
    \ruleno{Return}
  \]  
  
  \caption{Continuation-passing Style Semantics for Statements}
  \label{bs_cps}
\end{figure}

=======
\section{Procedures}

Procedures are unit-returning functions. We consider adding procedures as a separate step, since introducing full-fledged functions would require
essential redefinition of the semantics for expressions; at the same time the code generation for funictions is a little trickier, than for procedures, so
it is reasonable to split the implementation in two steps.

At the source level procedures are added as a separate syntactic category~--- definition $\mathscr D$:

\[
\begin{array}{rcl}
  \mathscr D & = & \epsilon \\
             &   & (\llang{fun $\;\mathscr X\;$ ($\mathscr X^*$) local $\;\mathscr X^*\;$ \{$\mathscr S$\}})\mathscr D
\end{array}
\]

In other words, a definition is a (possibly empty) sequence of procedure descriptions. Each description consists of a name for the procedure, a
list of names for its arguments and local variables, and a body (statement). In concrete syntax a single definition looks like

\begin{lstlisting}
  fun $name$ ($a_1$, $\;a_2$, $\dots$, $\;a_k$)
    local $l_1$, $\;l_2$, $\dots$, $\;l_n$ {
    $s$
  }  
\end{lstlisting}

where $name$~--- a name for the procedure, $a_i$~--- its arguments, $l_i$~--- local variables, $s$~--- body.

We also need to add a call statement to the language:

\[
\mathscr S += \mathscr X (\mathscr E^*)
\]

In a concrete syntax a call to a procedure $f$ with arguments $e_1,\dots,e_k$ looks like

\[
  \llang{$f\;$ ($e_1,\dots,e_k$)}
\]

Finally, we have to redefine the syntax category for programs at the top level:

\[
\mathscr L = \mathscr D\mathscr S
\]

In other words, we extend a statement with a set of definitions.

With procedures, we need to introduce the notion of \emph{scope}. When a procedure is called, its arguments
are associated with actual parameter values. A procedure is also in ``posession'' of its local variables.
So, in principle, the context of a procedure execution is a set of arguments, local variables and, possibly, some
other variables, for example, the global ones. However, the exact details of procedure context manipilation can
differ essentially from language to language.

In our case, we choose a static scoping~--- each procedure, besides its arguments and local variables, has an access
only to global variables.  To describe this semantics, we need to change the definition of a state we've used so far:

\[
\Sigma = (\mathscr X \to \mathbb Z) \times 2^{\mathscr X} \times (\mathscr X \to \mathbb Z)
\]

Now the state is a triple: a \emph{global} state, a set of variables, and a \emph{local} state. Informally, in a new
state $\inbr{\sigma_g,\,S,\,\sigma_l}$ $S$ describes a set of local variables, $\sigma_l$~--- their values, and $\sigma_g$~---
the values of all other accessible variables.

We need to redefine all state-manipulation primitives; first, the valuation of variables:

\[
\inbr{\sigma_g,\,S,\,\sigma_l}\;x=
  \left\{\begin{array}{rcl}
            \sigma_g\;x & , & x\not\in S\\
            \sigma_l\;x & , & x \in S
         \end{array}
  \right.
\]

Then, updating the state:

\[
\inbr{\sigma_g,\,S,\,\sigma_l}[x\gets z] =
  \left\{\begin{array}{rcl}
            \inbr{\sigma_g[x\gets z],\,S,\,\sigma_l} & , & x\not\in S\\
            \inbr{\sigma_g,\,S,\,\sigma_l[x\gets z]} & , & x \in S
         \end{array}
  \right.
\]

As an empty state, we take the following triple:

\[
\bot=\inbr{\bot,\,\emptyset,\,\bot}
\]

Finally, we need two transformations for states:

\[
\begin{array}{rcl}
  \mbox{\textbf{enter}}\,\inbr{\sigma_g,\,\_,\,\_}\,S & = & \inbr{\sigma_g,\,S,\,\bot}\\
  \mbox{\textbf{leave}}\,\inbr{\sigma_g,\,\_,\,\_}\,\inbr{\_,\,S,\,\sigma_l}& = & \inbr{\sigma_g,\,S,\,\sigma_l}
\end{array}
\]

The first one simlulates entering the new scope with a set of local variables $S$; the second one simulates leaving
from an inner scope (described by the first state) to an outer one (described by the second).

\setarrow{\xRightarrow}

All exising rules for big-step operational semantics have to be enriched by \emph{functional environment} $\Gamma$, which
binds procedure names to their definitions. As this binding never changes during the program interpretation, we need only to
propagate this environment, adding $\withenv{\Gamma}{...}$ for each transition ``$\trans{...}{}{...}$''. The only thing we
need now is to describe the rule for procedure calls:

\[
\trule{\withenv{\Gamma}{\trans{\inbr{\mbox{\textbf{enter}}\;\sigma\,(\bar{a}@\bar{l})[\overline{a\gets\sembr{e}\sigma}],\,i,\,o}}{S}{\inbr{\sigma^\prime,\,i^\prime,\,o^\prime}}}}
      {\withenv{\Gamma}{\trans{\inbr{\sigma,\,i,\,o}}{f (\bar{e})}{\inbr{\mbox{\textbf{leave}}\;\sigma^\prime\,\sigma,\,i^\prime,o^\prime}}}}
     \ruleno{Call$_{bs}$}
\]

where $\Gamma\,f = \llang{fun $\;f\;$ ($\bar{a}$) local $\;\bar{l}\;$ \{$S$\}}$.


\section{Extended Stack Machine}

In order to support procedures and calls, we enrich the stack machine with three following instructions:

\[
\begin{array}{rcl}
  \mathscr I & += & \llang{BEGIN $\;\mathscr X^*\;\mathscr X^*$} \\
             &    & \llang{CALL $\;\mathscr X$}\\
             &    & \llang{END}
\end{array}
\]

Informally speaking, instruction \llang{BEGIN} performs entering into the scope of a procedure; its operands are the lists of argument names and
local variables; \llang{END} leaves the scope and returns to the call site, and \llang{CALL} performs the call itself.

We need to enrich the configurations for the stack machine as well:

\[
\mathscr C_{SM} = (\mathscr P\times \Sigma)\times\mathbb Z^*\times \mathscr C
\]

Here we added a \emph{control stack}~--- a stack of pairs of programs and states. Informally, when performing the \llang{CALL} instruction, we put the following
program and current state on a stack to use them later on, when corresponding \llang{END} instruction will be encountered. As all other instructions does not
affect the control stack, it gets threaded through all rules of operational semantics unchanged.

Now we specify additional rules for the new instructions:

\[
\trule{\withenv{P}{\trans{\inbr{cs,\,st,\,\inbr{\mbox{\textbf{enter}}\;\sigma\,(\bar{a}@\bar{l})\overline{[a\gets z]},\,i,\,o}}}{p}{c^\prime}}}
      {\withenv{P}{\trans{\inbr{cs,\,\bar{z}@st,\,\inbr{\sigma,\,i,\,o}}}{(\llang{BEGIN $\;\bar{a}\,\bar{l}$})p}{c^\prime}}}
      \ruleno{Begin$_{SM}$}
\]

\[
\trule{\withenv{P}{\trans{\inbr{(p,\,\sigma)::cs,\,st,\,\inbr{\sigma,\,i,\,o}}}{P[f]}{c^\prime}}}
      {\withenv{P}{\trans{\inbr{cs,\,st,\,\inbr{\sigma,\,i,\,o}}}{(\llang{CALL $\;f$})p}{c^\prime}}}
      \ruleno{Call$_{SM}$}
\]

\[
\trule{\withenv{P}{\trans{\inbr{cs,\,st,\,\inbr{\mbox{\textbf{leave}}\;\sigma\,\sigma^\prime,\,i,\,o}}}{p^\prime}{c^\prime}}}
      {\withenv{P}{\trans{\inbr{(p^\prime,\,\sigma^\prime)::cs,\,st,\,\inbr{\sigma,\,i,\,o}}}{\llang{END}p}{c^\prime}}}
      \ruleno{EndRet$_{SM}$}
\]

\[
\withenv{P}{\trans{\inbr{\epsilon,\,st,\,c}}{\llang{END}p}{\inbr{\epsilon,\,st,\,c}}}\ruleno{EndStop$_{SM}$}
\]
>>>>>>> hw7
\end{document}
